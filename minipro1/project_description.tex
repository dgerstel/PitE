\documentclass[a4paper,10pt]{article}
\usepackage[utf8]{inputenc}

\usepackage{amsmath} 
% \usepackage[T1]{fontenc}
\usepackage{graphicx}
%\documentclass{article}
%\usepackage[showframe]{geometry}
\usepackage{layout}
\usepackage[margin=1.5in]{geometry}
%\usepackage{multirow}
\usepackage{hhline} %for double hlines
\usepackage{booktabs} %for all professional-looking table characterictics ;-)
\usepackage{float}
\restylefloat{table}
\floatstyle{plaintop}
\usepackage[tableposition=top]{caption}
\renewcommand{\arraystretch}{1.5}

\usepackage{subcaption}


\usepackage{xcolor}
\usepackage{listings}
\usepackage{caption}
\DeclareCaptionFont{white}{\color{white}}
\DeclareCaptionFormat{listing}{%
  \parbox{\textwidth}{\colorbox{gray}{\parbox{\textwidth}{#1#2#3}}\vskip-4pt}}
\captionsetup[lstlisting]{format=listing,labelfont=white,textfont=white}
\lstset{frame=lrb,xleftmargin=\fboxsep,xrightmargin=-\fboxsep}






\usepackage{listings}
\lstset{language=Python, showspaces=false, showstringspaces=false, numbers=left, breaklines=true}
% Title Page
\title{\textbf{Python in the enterprise} \\ mini-project \#1: Flight recorder simulator}
\author{Dawid Gerstel}

\begin{document}
\maketitle

The objective is to come up with an application recording flight parameters in real time. Additionally, the parameters should be accessible for future reference. 

\section{Flight simulation}
In order to generate necessary data the application user will move a mock aeroplane in 3 dimensions using keyboard with keys bound to respective control actions (acceleration, rotations in azimuthal and polar angles). The vehicle space orientation and motion will be displayed by simple gui, which together with keyboard readout will be implemented with the matplotlib library.

\section{Data storage}
The basic set of parameters taken into account consists of:
\begin{itemize}
    \item altitude
    \item longitude and latitude (or x and y) position
    \item aircraft orientation ($\varphi$ and $\theta$ angles)
    \item velocity (or, perhaps, merely speed)
    \item jet fuel status (consumption of which may be a function of altitude and acceleration).
\end{itemize}
These parameters will be displayed online in the gui (set of graphs) and will be recorded at fixed time intervals and written to a text file.
Apart from flight simulation the application shall provide functionality of displaying the data from previous flights (reading the data files and plotting their content).
After each flight session a report may be generated with such information like, for example, average speed, total fuel consumption, total distance travelled.



\end{document}          
